\documentclass[12pt,a4paper]{article}
\title{Note of Unix Programing}
\author{OTA Hsu}
\begin{document}
\maketitle
    \section{Vim}
        \begin{itemize}
        \item
        VIM is improved VI. GVIM is graphic VIM.
        \item
        VIM would initialize the settings in ~/.vimrc then ~/.gvimrc if they
        exist.
        \item
        In Win version, note that the setting of "map" for copy/paste and
        the setting of "behave xterm". 
        \end{itemize}
        \subsection{Start the application}
        In terminal, we can open gvim with multiple files in once.
        
        \$ gvim *.py -p
        \subsection{Set environment}
        The "set" command can be manipulated in ~/.vimrc, in command mode,
        or in the end of the text file.
        
        Ex. In the end of a .py file, we add 
        
        \# vim: set filetype=python fileformat=unix fileencoding=utf-8
        autoindent number shiftwidth=4 tabstop=4 extendtab textwidth=79.
        Where extendtab means that substituting 'tab' with many 'space',
        tabstop is the number of 'space', shiftwidth it the width of one
        indent that press '>>' in normal mode.
        \subsection{Normal mode}
        \begin{itemize}
        \item
        split window, 'Ctrl+w' then 'Ctrl+s'
        \item
        into Insert mode, 'i'
        \item
        into Replace mode, 'R'
        \item
        delete 3 lines right after cursor, '3dd'
        \item 
        yank copy 3 lines right after cursor, '3yy'
        \item
        \end{itemize}
        \subsection{Command mode}
        \subsection{Insert/Replace mode}
        \subsection{Visual/Select mode}


\end{document}

